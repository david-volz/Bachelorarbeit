\chapter{Introduction}
Data Security is a big issue for virtually every company. Physical access to server is restricted as well as connection to the corporate network from outside. However, most of the time it's the employees that really are dangerous for data integrity and safety. One big example is phishing. Many security measures are voided by employees getting tricked to give their password to outsiders. Today there are two new trends to jeopardize company networks. The first one is \textit{'Bring Your Own Device'} (BYOD), where employees work with their private end-user device within the corporate network. This makes it possible for anything on the employees device to access corporate assets. The second trend is the \textit{'Internet of Things'} (IoT). The idea of IoT is, that virtually every device/sensor is interconnected, which means, that IoT-devices within the corporate networks are able communicate with the outside world, creating a possible security issue. This thesis discuses what other issues for companies are created by BYOD and IoT and measures to profile and monitor any device in a corporate network.
